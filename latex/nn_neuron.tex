\documentclass[border=3pt,tikz]{standalone}
\usepackage{tikz}
\usepackage{kotex}
\usetikzlibrary{matrix,chains,positioning,decorations.pathreplacing,arrows}
\setmainfont[Mapping=tex-text]{나눔고딕}
\setsansfont[Mapping=tex-text]{나눔명조}
\setmonofont{나눔고딕코딩}


\begin{document}

\begin{tikzpicture}[
  init/.style={
    draw,
    circle,
    inner sep=2pt,
    font=\Huge,
    join = by -latex,
    x=1.6cm,y=1.1cm
  },
  squa/.style={
    draw,
    inner sep=2pt,
    font=\Large,
    join = by -latex
  },
  start chain=2,node distance=13mm
]
\node[on chain=2]
(x2) {$x_2$};
\node[on chain=2,join=by o-latex]
{$w_2$};
\node[on chain=2,init] (sigma)
{$\displaystyle\Sigma$};
\node[on chain=2,squa,label=above:{\parbox{2cm}{\centering 활성화 \\ 함수}}]
{$f$};
\node[on chain=2,label=above:{출력},join=by -latex]
{$y$};
\begin{scope}[start chain=1]
\node[on chain=1] at (0,1.5cm)
(x1) {$x_1$};
\node[on chain=1,join=by o-latex]
(w1) {$w_1$};
\end{scope}
\begin{scope}[start chain=3]
\node[on chain=3] at (0,-1.5cm)
(x3) {$x_3$};
\node[on chain=3,label=below:{가중치},join=by o-latex]
(w3) {$w_3$};
\end{scope}
\node[label=above:\parbox{2cm}{\centering 편향 \\ $b$}] at (sigma|-w1) (b) {};

\draw[-latex] (w1) -- (sigma);
\draw[-latex] (w3) -- (sigma);
\draw[o-latex] (b) -- (sigma);

\draw[decorate,decoration={brace,mirror}] (x1.north west) -- node[left=10pt] {입력} (x3.south west);
\end{tikzpicture}

\end{document}
